\documentclass[a4paper]{article}
\usepackage{longtable}
\usepackage[color]{vdmlisting}
\usepackage{fullpage}
\usepackage{hyperref}
\begin{document}
\title{}
\author{}
\begin{vdm_al}
module VELOCITY
imports
from VDMUtil
 functions
  val2seq_of_char renamed val2seq_of_char;
,
from Support1
 types 
  String renamed String;
  String1 renamed String1;
 functions
  fold renamed fold;
  prods_r renamed prods_r;
  averages_r renamed averages_r;
  replicate renamed replicate;
exports 
  types
    --@todo which ones to struct export which not to?
    struct Dimension;
  struct DimensionlessVector;
  struct DimensionVector;
  struct DimensionView;
  struct SingleDimension;
  struct Magnitude;
  struct MagnitudeN0;
  struct Magnitude1;
  struct Quantity;
  struct QuantityN0;
  struct SQuantity;
  struct SQuantityN0;
  struct ConversionSchema;

  --@todo make use of those
    struct Conversion;
  struct Conversions;
        
  struct MeasurementSystem; 
  struct MeasurementSystemN0;
  struct Prefix; 

  struct Second;
  
    struct SI_MeasurementSystem;
  struct Meter;
  struct Kilogram;
  struct Kelvin;

    struct Frequency;
    struct Velocity;

  struct BIS_MeasurementSystem;
  struct Yard;
  struct Pound;
  struct Rankine;
  struct BISVelocity;

  struct CGS_MeasurementSystem;
  struct Centimetre;
  struct Gram;
  
  struct MHC_MeasurementSystem;
  struct Milligram;
  struct Hour;
  struct Celcius;

  struct WhichMonth;
  struct HowManyDays;
 
  struct MDC_MeasurementSystem;
  struct Day;
  
  struct MWC_MeasurementSystem;
  struct Week;
   
  functions
    dim_comp: DimensionlessVector * DimensionlessVector -> DimensionlessVector;
  dim_comp_n: DimensionlessVector * nat -> DimensionlessVector;
  dim_inv: DimensionlessVector -> DimensionlessVector;
  dim_inv_n: DimensionlessVector * nat1 -> DimensionlessVector;
  dim_div: DimensionlessVector * DimensionlessVector -> DimensionlessVector;
  dim_view: DimensionView * DimensionlessVector -> String1;
  si_dim_view: Prefix -> String1;
  bis_dim_view: Prefix -> String1;
  
  quant_dim_eq: Quantity * Quantity -> bool;
  quant_times: Quantity * Quantity -> Quantity;
  quant_div: Quantity * QuantityN0 -> Quantity;
  quant_inv: QuantityN0 -> Quantity;
  quant_itself_n: Quantity * nat1 -> Quantity;
  quant_inv_n: QuantityN0 * nat1 -> Quantity;
  quant_plus: Quantity * Quantity -> Quantity;
  quant_uminus: Quantity -> Quantity;
  quant_minus: Quantity * Quantity -> Quantity;
  scaleQ: Magnitude * Quantity -> Quantity;
 
  scaleMS: Magnitude * MeasurementSystem -> MeasurementSystem;
  quant_conv: ConversionSchema * DimensionlessVector -> MagnitudeN0;
  conv_inv: ConversionSchema -> ConversionSchema;
  conv_comp: ConversionSchema * ConversionSchema -> ConversionSchema;
  ms_conv_eq: MeasurementSystem * MeasurementSystem -> bool;
  ms_times: MeasurementSystem * MeasurementSystem -> MeasurementSystem;
  ms_itself_n: MeasurementSystem * nat1 -> MeasurementSystem;
  ms_inv_n: MeasurementSystemN0 * nat1 -> MeasurementSystem;
  ms_inv: MeasurementSystemN0 -> MeasurementSystem;
  ms_div: MeasurementSystem * MeasurementSystemN0 -> MeasurementSystem;
   
  ms_quant_conv: ConversionSchema * MeasurementSystem -> MeasurementSystem;
  ms_conv: /*ConversionSchema * ConversionSchema * MeasurementSystem*/
       MeasurementSystem * ConversionSchema -> MeasurementSystem;
 
  mag  : Prefix -> Magnitude;
  deca : Prefix -> Prefix;
  hecto: Prefix -> Prefix;
  kilo : Prefix -> Prefix;
  mega : Prefix -> Prefix;
  giga : Prefix -> Prefix;
  tera : Prefix -> Prefix;
  deci : Prefix -> Prefix;
  centi: Prefix -> Prefix;
  milli: Prefix -> Prefix;
  micro: Prefix -> Prefix;
  nano : Prefix -> Prefix;
  pico : Prefix -> Prefix;
  
  second: ConversionSchema * UnitSystem -> Second;
  minute: ConversionSchema * UnitSystem -> Second;
  hour  : ConversionSchema * UnitSystem -> Second;
  day   : ConversionSchema * UnitSystem -> Second;
  week  : ConversionSchema * UnitSystem -> Second;

    metrify: MeasurementSystem -> Quantity;
  SI_ASTRONOMICAL     : () -> Meter;
 
  SI_YARD             : () -> Meter;
  SI_FOOT             : () -> Meter;
  SI_INCH             : () -> Meter;
  SI_MILE             : () -> Meter;
  SI_NAUTICAL_MILE    : () -> Meter;
  
  SI_TONNE            : () -> Kilogram; 
  SI_POUND            : () -> Kilogram;
  SI_OUNCE            : () -> Kilogram;
  SI_STONE            : () -> Kilogram;
  
  SI_KNOT             : () -> Velocity;
  SI_MPH              : () -> Velocity;
  SI_MPS              : () -> Velocity;
  SI_KPH              : () -> Velocity;
  SI_DEGREES_FARENHEIT: Magnitude -> Kelvin;

    imperialise: MeasurementSystem -> Quantity;
  BIS_FOOT         : () -> Yard;
  BIS_INCH         : () -> Yard;
  BIS_MILE         : () -> Yard;
 
  BIS_OUNCE        : () -> Pound;
  BIS_MILE_PER_HOUR: () -> BISVelocity;
 
  hDAY      : () -> Hour;
  hWEEK     : () -> Hour;
  hMONTH    : WhichMonth -> Hour; 
  hAVG_MONTH: () -> Hour;
  hYEAR     : () -> Hour;
  weeks2months_hrs: nat1 -> Hour;
  n_times_day2every_x_hours: nat1 -> real;
     
  dWEEK: () -> Day;
  weeks2days: nat1 -> Day;
 
  wDAY      : () -> Week;
  wAVG_MONTH: () -> Week;
  weeks2months_wks: nat1 -> Week;
 
  approx: Magnitude * nat -> Magnitude;
  approx_eq: Magnitude * Magnitude * nat -> bool;
  mph2mps: Magnitude -> Magnitude;
 
  MEASUREMENT_SYSTEMS: () -> set1 of MeasurementSystem; 
  CONVERSION_SCHEMAS : () -> set1 of ConversionSchema;
  DIMENSION_VECTORS  : () -> set1 of DimensionVector;

  values
   DIMENSIONS       : set1 of Dimension;
  ZERO_DV          : DimensionlessVector;
  DLENGTH          : SingleDimension;
  DMASS            : SingleDimension;
  DTIME            : SingleDimension;
  DTEMP            : SingleDimension;
                   
  DFREQUENCY       : DimensionVector; 
  DVELOCITY        : DimensionVector; 
 
  SI_DIM_VIEW   : DimensionView;
  BIS_DIM_VIEW  : DimensionView;
                    
  CONV_ID          : ConversionSchema;
  
  UNIT_LENGTH      : IntQuantity; 
  UNIT_MASS        : IntQuantity; 
  UNIT_TIME        : IntQuantity; 
  UNIT_TEMP        : IntQuantity; 

    UNIT_FREQUENCY   : Quantity;
  UNIT_VELOCITY    : Quantity;
 
    SI            : ConversionSchema;   
  METER            : Meter   ;
  KILOGRAM         : Kilogram;
  SECOND          : Second  ;
  KELVIN           : Kelvin  ;
   SI_FREQUENCY     : Frequency   ;
  SI_VELOCITY      : Velocity    ;

  BIS           : ConversionSchema;
  BIS_YARD      : Yard    ;
  BIS_POUND        : Pound   ;
    BIS_VELOCITY  : BISVelocity;
  BIS_RANKINE      : Rankine ;

    CGS              : ConversionSchema;
  CENTIMETRE       : Centimetre;
  GRAM             : Gram   ;

    MHC              : ConversionSchema;
  MGRAM       : Milligram;
  MHOUR            : Hour   ; 
  MCELCIUS         : Celcius ; 

  
  MDC     : ConversionSchema;
  MDAY     : Day;
                      
  MWC        : ConversionSchema;
  MWEEK       : Week;
 
  HOURS_PER_DAY    : nat1;
  DAYS_PER_WEEK    : nat1;
  WEEKS_PER_YEAR   : nat1;
  MONTHS_PER_YEAR  : nat1;
  DAYS_PER_MONTH   : map WhichMonth to HowManyDays;
 
    HERTZ    : Hertz;
  CAESIUM_FREQUENCY: Hertz;
  SPEED_OF_LIGHT   : Velocity;
definitions

------------------------------------------------------------------------------------
--@chapter International Standard Units (and Quantities) 
------------------------------------------------------------------------------------
types

------------------------------------------------------------------------------------
--@section ISQ dimensions
------------------------------------------------------------------------------------

 --@doc SI base units, ordered according to SI's listing; ordered them for pretty printing only
 Dimension = <Length> | <Mass> | <Time> | <Temperature>
 ord d1 < d2 == 
  let 
   value = { <Length> |-> 1, <Mass> |-> 2, <Time> |-> 3,  <Temperature> |-> 4
         }
    in value(d1) < value(d2);
    
values
 DIMENSIONS: set1 of Dimension = {<Length>, <Mass>, <Time>, <Temperature>};

types
  --@doc SI dimension compositions: two operators, times (counts quantity) and inverse (counts negative quantity)
 --   total mapping to ensure that we have full dimension information for all base units of interst 
 --   ex: pressure as kg/ms^2 = { <Mass> |-> 1, <Length> |-> -1, <Time> |-> -2 } (i.e. pascal = kilo per metre per second per second)  
 --   ex: velocity as m/s = { <Length> |-> 1, <Timne> |-> -2 }
 DimensionVector0= map Dimension to int;
 DimensionlessVector = DimensionVector0
 inv dv == 
  --@doc needs info on all known dimensions
  --@OnFail(4060, "Dimensionless vector missing %1s dimensions = %2s", DIMENSIONS \ dom dv, dv)
  (dom dv = DIMENSIONS)
  and
  --@doc needs at least one dimension set 
  true--rng(dv :-> {0}) <> {} 
  --(dunion rng dv) \ {0} <> {}
  ;

 --@doc Invariant insits that there is some dimension in the vector. 
 --     For dimensionless, we get the unit vector (algebraic one), as in DRADIAN or DSTERADIAN
 DimensionVector = DimensionlessVector
 inv dv == 
  --OnFail(4060, "Dimension vector has only dimensionless (0) ranges for %1s", dom(dv :> {0}))
  (rng(dv :-> {0}) <> {});
        
 --@doc The vector where only one of the base unit dimension registers 
 --    (i.e. only one mapping to 1, and all other mappings are to zero)
 SingleDimension = DimensionVector
 inv sd == card dom (sd :> {1}) = 1 and rng (sd :-> {1}) = {0};
  
------------------------------------------------------------------------------------
--@section ISQ quantities
------------------------------------------------------------------------------------
    
  --@doc magnitude values 
 Magnitude   = real;
 MagnitudeN0 = Magnitude inv m == m <> 0;
 Magnitude1  = Magnitude inv m == m > 0;

 --@doc quantity semantic domain as magnitude and a dimension vector to measure it within (e.g. 3 km/hr)
 --@doc the domain is ordered by magnitude when within the same dimension; see quant_lt operators below 
 --@doc some quantities are dimensionless (e.g. radian is meter/meter)
 --   (i.e. can't compare different dimension quantities directly)
 --@doc quantity equality only compares magnitude.  
 Quantity ::
  mag: Magnitude
  dim:- DimensionlessVector
 --eq  mk_Quantity(m1, d1) = mk_Quantity(m2, d2) == m1 = m2 --and d1 = d2
 ord mk_Quantity(m1, -) < mk_Quantity(m2, -) == m1 < m2 ; --and d1 = d2;
        
 --@doc non-zero quantity in any dimension (e.g. x dimension, x > 0)
 QuantityN0 = Quantity
 inv mk_Quantity(m, -) == is_MagnitudeN0(m);

 --@doc single dimension quantity (e.g. 3km, no 3km/hr)
 SQuantity = Quantity
 inv mk_Quantity(-, d) == is_SingleDimension(d);

 --@doc non-zero single dimension quantity
 SQuantityN0 = SQuantity
 inv sq == is_QuantityN0(sq);
        
 --@doc integer (whole) quanity
 IntQuantity = Quantity
 inv iq == is_int(iq.mag);
            
------------------------------------------------------------------------------------
--@section ISQ measurement systems
------------------------------------------------------------------------------------

 --@doc the name of unit system for measurement; could be enums, but then harder to extend?
 UnitSystem = String1;
        
 --@doc a conversion schema defines (non zero magnitude) factors to convert between 
 --    measurement systems for all standard dimensions
 ConversionSchema = map Dimension to MagnitudeN0
 inv cs == dom cs = DIMENSIONS;
        
 --@doc Each conversion target associates different conversion schemas to a specific unit system target
 --    (e.g. converting to BIS with given schema from whatever unit) 
 --@doc Conversions Source to Target unit system conversion schemas.
 Conversion  = map UnitSystem to ConversionSchema;
        
 --@todo conversions matrix could be useful for well known transformations of interest to be instantiated.
 Conversions = map UnitSystem to Conversion
 --@doc It's always possible to convert to itself: crucial to allow bottoming constants up 
 inv cs == forall u in set dom cs & u in set dom cs(u); -- dom cs = dunion { dom i | i in set (rng cs) }
        
 --@doc a measurement system is a conversion schema and a typed quantity, possibly in multiple dimensions
 --@doc equality and comparison can happen between mulitple dimensions within the same measurement system
 --    (i.e. if conversion schemas are the same for the given unit system)
 MeasurementSystem ::
  quantity: Quantity
   schema: ConversionSchema
    unit: UnitSystem
  --@doc quantity dimension vector must agree with conversion schema (e.g. know what to convert to)
 inv mk_MeasurementSystem(mk_Quantity(-, d), s, -) == dom d = dom s
 eq  mk_MeasurementSystem(q1, s1, u1) 
  = 
  mk_MeasurementSystem(q2, s2, u2) 
  ==
  q1 = q2 and s1 = s2 and u1 = u2
 ord mk_MeasurementSystem(mk_Quantity(m1, -), -, -) 
  < 
  mk_MeasurementSystem(mk_Quantity(m2, -), -, -) 
  ==
  --OnFail(4087, "Cannot compare measurement systems: %1s(%2s) < %3s(%4s)?", u1, m1, u2, m2)
  (m1 < m2)-- and s1 = s2 and u1 = u2)
 ;

 MeasurementSystemN0 = MeasurementSystem
 inv ms == is_QuantityN0(ms.quantity);

 --@doc we can prefix either magnitude, quantity or measurement system
 Prefix = (Magnitude | Quantity | MeasurementSystem);

values
 --@doc the dimensionless vector is the one in the vector algebra (i.e. dimensions are added/substracted)
 ZERO_DV   : DimensionlessVector= { i |-> 0 | i in set DIMENSIONS };
        
functions   
------------------------------------------------------------------------------------
--@section ISQ dimension operators
------------------------------------------------------------------------------------

 --@doc composing dimentions is aking to adding them (i.e. exponent laws: metre**i * metre**j = metre**(i+j))
 --   (e.g. watt = kg * m**2 / s**2, so L=2,M=1,T=-2)
 dim_comp: DimensionlessVector * DimensionlessVector -> DimensionlessVector
 dim_comp(d1, d2) == { d |-> d1(d) + d2(d) | d in set dom d1 inter dom d2 }
 post 
  --@doc if result has dimension, it must have come from either d1 or d2
  --@doc can be dimensionless for case of pure quantities (e.g. radian)
  true--(is_DimensionVector(RESULT) <=> is_DimensionVector(d1) or is_DimensionVector(d2))
  ;
            
  --@doc utility for d**2 or d**3 etc (i.e. compose twice/thrice with itself)
  dim_comp_n: DimensionlessVector * nat -> DimensionlessVector
  dim_comp_n(di, n) == 
   fold[DimensionlessVector](dim_comp, ZERO_DV, replicate[DimensionVector](n, di)) 
 post 
  --@doc if result has dimension, it must have come from either d1 or d2
  --@doc can be dimensionless for case of pure quantities (e.g. radian)
  true--(is_DimensionVector(RESULT) <=> is_DimensionVector(di))
 measure
  n;
            
 --@doc inverting dimensions is aking to negating them (i.e. exponent law: metre**-i * 1 / metre**i)
 dim_inv: DimensionlessVector -> DimensionlessVector
 dim_inv(di) == { d |-> -di(d) | d in set dom di }
 post 
  --@doc if result has dimension, it must have come from either d1 or d2
  --@doc can be dimensionless for case of pure quantities (e.g. radian)
  true;--(is_DimensionVector(RESULT) <=> is_DimensionVector(di));
        
 --@doc utility for d**-2 or d**-3 etc (i.e. compose twice/thrice with itself then invert)
 --@todo remove for dim_comp_n negative calling inv?
 dim_inv_n: DimensionlessVector * nat1 -> DimensionlessVector
 dim_inv_n(di, n) == dim_inv(dim_comp_n(di, n))
 post 
  --@doc if result has dimension, it must have come from either d1 or d2
  true;--(is_DimensionVector(RESULT) <=> is_DimensionVector(di));

 --@doc dividing dimensions is aking to composing with the second dimension's inverse 
 dim_div: DimensionlessVector * DimensionlessVector -> DimensionlessVector
 dim_div(d1, d2) == dim_comp(d1, dim_inv(d2))
 post 
  --@doc if result has dimension, it must have come from either d1 or d2
  true;--(is_DimensionVector(RESULT) <=> is_DimensionVector(d1) or is_DimensionVector(d2));
        
types 
 --@doc human readable string for a dimension vector
 DimensionView = map Dimension to String1
 inv dv == dom dv = DIMENSIONS;

values 
 --@doc specific dimension view for each measurement system
 SI_DIM_VIEW: DimensionView = 
  {<Length> |-> "m", <Mass> |-> "kg", <Time> |-> "s" , <Temperature> |-> "K"};

 BIS_DIM_VIEW: DimensionView = SI_DIM_VIEW ++ 
  {<Length> |-> "yard", <Mass> |-> "pound", <Temperature> |-> "rankine" };

functions

 --@doc mapps given dimension on dimension vector to its corresponding dimension view
 print_dim: Dimension * DimensionView * DimensionVector -> String1 
 print_dim(d, dview, dvector) ==
  let exp: nat1 = abs dvector(d) in
   if exp <= 1 then 
    dview(d)
   else 
    "(" ^ dview(d) ^ "**" ^ val2seq_of_char[nat1](exp) ^ ")"  
 pre dvector(d) <> 0;

 --@doc creates a dimension vector print out  
 print_dims: set of Dimension * DimensionView * DimensionVector -> String 
 print_dims(dims, dview, dvector) ==
  if dims = {} then 
   ""
  else 
   let d in set dims in
    print_dim(d, dview, dvector) ^ " " ^ print_dims(dims \ {d}, dview, dvector)
 pre 
  --@doc cannot print dimensionless dimension!
  not 0 in set rng (dims <: dvector)
 post 
  --@doc results are non empty except at the end
  (card dims <> 0 => is_String1(RESULT))
 measure
  card dims;

 parenthesise: bool * String -> String1
 parenthesise(y, s) == if y then "( " ^ s ^ " )" else s;

 --@doc user function to prints a dimention's vector view 
 dim_view: DimensionView * DimensionlessVector -> String1
 dim_view(dview, dvector) == 
  if not is_DimensionVector(dvector) then 
   "1"
  else 
   let 
    pos_dimensions: set of Dimension = { d | d in set dom dvector & dvector(d) > 0 },
    neg_dimensions: set of Dimension = { d | d in set dom dvector & dvector(d) < 0 },
    nominator  : String = print_dims(pos_dimensions, dview, dvector),
    denominator: String = print_dims(neg_dimensions, dview, dvector)  
   in 
    (if pos_dimensions = {} then 
     " 1 " 
        else 
      parenthesise(neg_dimensions <> {} and card pos_dimensions > 1, nominator) 
    )
    ^ 
    (if neg_dimensions <> {} then 
     " / " ^ parenthesise(card neg_dimensions > 1, denominator) 
     else 
        ""
    );

 --@doc dimension view for any SI prefix
 si_dim_view: Prefix -> String1 
 si_dim_view(x) == 
  cases true:
   (is_Magnitude(x))         -> val2seq_of_char[Magnitude](x),
   (is_Quantity(x))          -> dim_view(SI_DIM_VIEW, x.dim),
   (is_MeasurementSystem(x)) -> dim_view(SI_DIM_VIEW, x.quantity.dim)
  end;

 --@doc dimension view for any BIS prefix
 bis_dim_view: Prefix -> String1 
 bis_dim_view(x) == 
  cases true:
   (is_Magnitude(x))         -> val2seq_of_char[Magnitude](x),
   (is_Quantity(x))          -> dim_view(BIS_DIM_VIEW, x.dim),
   (is_MeasurementSystem(x)) -> dim_view(BIS_DIM_VIEW, x.quantity.dim)
  end;

------------------------------------------------------------------------------------
--@section ISQ quantity operators
------------------------------------------------------------------------------------
        
 --@doc check whether two dimensions have the same quantity dimension vector
 quant_dim_eq: Quantity * Quantity -> bool
 quant_dim_eq(mk_Quantity(-, d1), mk_Quantity(-, d2)) == d1 = d2; 
            
 --@doc multiplying quantities multiply their magnitude and compose their dimensions
 quant_times: Quantity * Quantity -> Quantity
 quant_times(mk_Quantity(m1, d1), mk_Quantity(m2, d2)) == 
  mk_Quantity(m1*m2, dim_comp(d1, d2));
            
 --@doc given m, 3 you get m**3
 quant_itself_n: Quantity * nat1 -> Quantity
 quant_itself_n(mk_Quantity(m, d), n) == mk_Quantity(m, dim_comp_n(d, n));
            
 --@doc dividing quantities divides their magnitude and divides their dimensions
 --@doc notice the second argument must be a non-zero quantity
 quant_div: Quantity * QuantityN0 -> Quantity
 quant_div(mk_Quantity(m1, d1), mk_Quantity(m2, d2)) ==
  mk_Quantity(m1/m2, dim_div(d1, d2));
            
 --@doc given q, you get 1/q 
 --@doc inverting quantities invert their magnitude and invert their dimensions
 quant_inv: QuantityN0 -> Quantity
 quant_inv(q) == quant_div(mk_Quantity(1, q.dim), q);

 --@doc given q, you get 1/q**n 
 quant_inv_n: QuantityN0 * nat1 -> Quantity
 quant_inv_n(q, n) == quant_inv(quant_itself_n(q, n));
            
 --@doc summing quantities sum their magnitude, providing they have the same dimension
 --@doc if it's for a single dimension quantity input, you get a single dimension quantity output 
 quant_plus: Quantity * Quantity -> Quantity
 quant_plus(d1, d2) == mk_Quantity(d1.mag + d2.mag, d1.dim)
 pre
  quant_dim_eq(d1, d2)
 post
  (is_SQuantity(d1) <=> is_SQuantity(RESULT));
            
 --@doc summing quantities sum their magnitude, providing they have the same dimension
 quant_uminus: Quantity -> Quantity
 quant_uminus(d1) == mk_Quantity(-d1.mag, d1.dim)
 post
  (is_SQuantity(d1) <=> is_SQuantity(RESULT));

 --@doc summing quantities sum their magnitude, providing they have the same dimension
 quant_minus: Quantity * Quantity -> Quantity
 quant_minus(d1, d2) == mk_Quantity(d1.mag - d2.mag, d1.dim)
 pre
  quant_dim_eq(d1, d2)
 post
  (is_SQuantity(d1) <=> is_SQuantity(RESULT));

 --@doc scale a quantity magnitude by given magnitude 
 scaleQ: Magnitude * Quantity -> Quantity
 scaleQ(m1, mk_Quantity(m2, d)) == mk_Quantity(m1 * m2, d);

------------------------------------------------------------------------------------
--@section ISQ basic dimension vectors and common units
------------------------------------------------------------------------------------

values
 --@doc basic dimension vectors
 DLENGTH   : SingleDimension = ZERO_DV ++ { <Length>      |-> 1 };
 DMASS     : SingleDimension = ZERO_DV ++ { <Mass>        |-> 1 };
 DTIME     : SingleDimension = ZERO_DV ++ { <Time>        |-> 1 };
 DTEMP     : SingleDimension = ZERO_DV ++ { <Temperature> |-> 1 };

 --@doc common dimension units
 DFREQUENCY   : DimensionVector = dim_inv(DTIME);                                          --T**-1
 DVELOCITY    : DimensionVector = dim_comp(DLENGTH, DFREQUENCY);                           --L*T**-1
        
 --@doc identity conversion schema maps all dimensions to 1 (i.e. no conversion)
 CONV_ID     : ConversionSchema = { i |-> 1 | i in set DIMENSIONS };
    
------------------------------------------------------------------------------------
--@section International Standard Units constants basic, accepted, derived, prefixes 
------------------------------------------------------------------------------------

 --@doc single unit of quantity for each base dimension  
 UNIT_LENGTH   : IntQuantity   = mk_Quantity(1, DLENGTH);
 UNIT_MASS     : IntQuantity   = mk_Quantity(1, DMASS);
 UNIT_TIME     : IntQuantity   = mk_Quantity(1, DTIME);
 UNIT_TEMP     : IntQuantity   = mk_Quantity(1, DTEMP);

 --@doc common SI prefixes
 PREFIX_DECA   : MagnitudeN0 = (10**1);
 PREFIX_HECTO  : MagnitudeN0 = (10**2)  ;
 PREFIX_KILO   : MagnitudeN0 = (10**3)  ;
 PREFIX_MEGA   : MagnitudeN0 = (10**4)  ;
 PREFIX_GIGA   : MagnitudeN0 = (10**5)  ;
 PREFIX_TERA   : MagnitudeN0 = (10**6);
 PREFIX_DECI   : MagnitudeN0 = 1/PREFIX_DECA;
 PREFIX_CENTI  : MagnitudeN0 = 1/PREFIX_HECTO;
 PREFIX_MILLI  : MagnitudeN0 = 1/PREFIX_KILO;
 PREFIX_MICRO  : MagnitudeN0 = 1/PREFIX_MEGA;
 PREFIX_NANO   : MagnitudeN0 = 1/PREFIX_GIGA;
 PREFIX_PICO   : MagnitudeN0 = 1/PREFIX_TERA;

 --@doc single unit of quantity for each base dimension  

 UNIT_FREQUENCY   : IntQuantity = mk_Quantity(1, DFREQUENCY)    ;
 UNIT_VELOCITY    : IntQuantity = mk_Quantity(1, DVELOCITY)     ;

------------------------------------------------------------------------------------
--@section SI basic and derived unit types
------------------------------------------------------------------------------------
 SI_UNIT: UnitSystem = "SI";
 
 --@doc International Systems of Units "Systeme Internacional" base units and conversion schemas
 SI: ConversionSchema = CONV_ID;

 HOURS_PER_DAY: nat1 = 24;
 DAYS_PER_WEEK: nat1 =  7;

types
 --@doc an SI measurement system has an SI conversion
 SI_MeasurementSystem = MeasurementSystem
 inv mk_MeasurementSystem(-, s, u) == s = SI and u = SI_UNIT;
        
 --@doc every standard basic dimension has an SI measurement system type
 Meter = SI_MeasurementSystem
 inv ms == ms.quantity.dim = DLENGTH;

 Kilogram = SI_MeasurementSystem
 inv ms == ms.quantity.dim = DMASS;

 --@doc second is common between different conversion schemas, so only fix the dimension, not conversion schema
 Second = MeasurementSystem
 inv ms == ms.quantity.dim = DTIME;
 
 Kelvin = SI_MeasurementSystem
 inv ms == ms.quantity.dim = DTEMP;
 
 Frequency = SI_MeasurementSystem
 inv ms == ms.quantity.dim = DFREQUENCY;

 Velocity = SI_MeasurementSystem
 inv ms == ms.quantity.dim = DVELOCITY;

 --@doc some commonly derived SI measurement system dimensions types
 Hertz = Frequency;

functions 
------------------------------------------------------------------------------------
--@section ISQ scaling and conversion over quantities and measurement systems
------------------------------------------------------------------------------------

 --@doc scale a dimension by given magnitude within a measurement system
 scaleMS: Magnitude * MeasurementSystem -> MeasurementSystem
 scaleMS(m1, mk_MeasurementSystem(q, s, u)) == mk_MeasurementSystem(scaleQ(m1, q), s, u);
 
 --@doc to quantity convert in multiple dimensions, we must use product of the integer exponenciation on the dimensions
 --     for all dimensions. Those with zero dimension, will lead to 1, others will be multiplied accordingly
 quant_conv: ConversionSchema * DimensionlessVector -> MagnitudeN0
 quant_conv(cs, dv) ==
  prods_r({ cs(i)**dv(i) | i in set dom cs inter dom dv })
 pre
  dom cs = dom dv;

------------------------------------------------------------------------------------
--@section ISQ measurement systems operators
------------------------------------------------------------------------------------
 
 --@doc check whether two measurement systems have the same unit system
 ms_conv_eq: MeasurementSystem * MeasurementSystem -> bool
 ms_conv_eq(m1, m2) == m1.schema = m2.schema and m1.unit = m2.unit;
  
 --@doc multiplying measurement system quantities providing we keep their unit system
 ms_times: MeasurementSystem * MeasurementSystem -> MeasurementSystem
 ms_times(m1, m2) == 
  mk_MeasurementSystem(quant_times(m1.quantity, m2.quantity), m1.schema, m1.unit)
 pre
  ms_conv_eq(m1, m2);
 
 ms_itself_n: MeasurementSystem * nat1 -> MeasurementSystem
 ms_itself_n(mk_MeasurementSystem(q, s, u), n) == 
  mk_MeasurementSystem(quant_itself_n(q, n), s, u);  

 --@doc inverting measurement system quantities providing we keep their conversion scehma
 ms_inv: MeasurementSystemN0 -> MeasurementSystem
 ms_inv(mk_MeasurementSystem(q, s, u)) == mk_MeasurementSystem(quant_inv(q), s, u);
  
 ms_inv_n: MeasurementSystemN0 * nat1 -> MeasurementSystem
 ms_inv_n(m, n) == ms_inv(ms_itself_n(m, n));
  
 --@doc dividing quantities divides their magnitude and divides their dimensions
 --@doc notice the second argument must be a non-zero quantity
 ms_div: MeasurementSystem * MeasurementSystemN0 -> MeasurementSystem
 ms_div(m1, m2) == mk_MeasurementSystem(quant_div(m1.quantity, m2.quantity), m1.schema, m1.unit)
 pre 
  ms_conv_eq(m1, m2);
   
------------------------------------------------------------------------------------
--@section ISQ conversion operators
------------------------------------------------------------------------------------
 
 --@doc inverse of conversion  
 conv_inv: ConversionSchema -> ConversionSchema
 conv_inv(cs) == { i |-> 1/cs(i) | i in set dom cs }
 post
  dom cs = dom RESULT;
 
 --@doc composition (times) conversion
 conv_comp: ConversionSchema * ConversionSchema -> ConversionSchema
 conv_comp(cs1, cs2) == { i |-> cs1(i)*cs2(i) | i in set dom cs1 inter dom cs2 }
 pre
  dom cs1 = dom cs2
 post
  dom cs1 = dom RESULT;

 --@doc given a conversion schema (cs_conv) and a measurement system, convert the quantity by 
 --     converting the magnitude according to the schema (cs_conv), keeping the given schema (cs)
 --     (i.e. it is will be transformed by ms_conv)
 ms_quant_conv: ConversionSchema * MeasurementSystem -> MeasurementSystem
 ms_quant_conv(cs_conv, mk_MeasurementSystem(mk_Quantity(m, d), s, u)) ==
  --@todo this ought to be cs_conv unit for s?!
  mk_MeasurementSystem(mk_Quantity(quant_conv(cs_conv, d) * m, d), s, u);
  
 --@doc From -> To convert(ms): 
 --     given a measurement system (ms), convert its quantity from the given conversion schema 
 --     to the selected conversion schema (i.e. inverting and composing schemas, then
 --     converting the magnitude accordingly). Avoid id conversion for speed.
 ms_conv: /*ConversionSchema * ConversionSchema * MeasurementSystem*/ MeasurementSystem * ConversionSchema -> MeasurementSystem
 ms_conv(/*cs_from, cs_to, ms*/ms, cs_to) == 
  let
   cs_from : ConversionSchema = ms.schema
   in
   if cs_from = cs_to then 
    ms
   else
    ms_quant_conv(conv_comp(cs_from, conv_inv(cs_to)), ms)
 /*
 --@todo this is to take Conversions into account, which isn't there yet? 
 pre
  cs_from = ms.schema
 post
  conv_comp(cs_from, conv_inv(cs_to)) = RESULT.schema;
 */ 
 ;

------------------------------------------------------------------------------------
--@section SI prefixes
------------------------------------------------------------------------------------
 
 --@doc project the magnitude of a quantity or measurement system
 mag: Prefix -> Magnitude
 mag(x) == 
  cases true:
   (is_Magnitude(x))         -> x,
   (is_Quantity(x))          -> x.mag,
   (is_MeasurementSystem(x)) -> x.quantity.mag
  end;

 --@doc multiply magnitude, quantity or measurement system
 scale_prefix: Prefix * Magnitude -> Prefix
 scale_prefix(x, p) == 
  cases true:
   (is_Magnitude(x))         -> x * p,
   (is_Quantity(x))          -> scaleQ(p, x),
   (is_MeasurementSystem(x)) -> scaleMS(p, x)
  end;

 deca: Prefix -> Prefix
 deca(x) == scale_prefix(x, PREFIX_DECA );

 hecto: Prefix -> Prefix
 hecto(x)== scale_prefix(x, PREFIX_HECTO);

 kilo: Prefix -> Prefix
 kilo(x) == scale_prefix(x, PREFIX_KILO );

 mega: Prefix -> Prefix
 mega(x) == scale_prefix(x, PREFIX_MEGA );

 giga: Prefix -> Prefix
 giga(x) == scale_prefix(x, PREFIX_GIGA );

 tera: Prefix -> Prefix
 tera(x) == scale_prefix(x, PREFIX_TERA );
 
 deci: Prefix -> Prefix
 deci(x) == scale_prefix(x, PREFIX_DECI );

 centi: Prefix -> Prefix
 centi(x)== scale_prefix(x, PREFIX_CENTI);

 milli: Prefix -> Prefix
 milli(x)== scale_prefix(x, PREFIX_MILLI);

 micro: Prefix -> Prefix
 micro(x)== scale_prefix(x, PREFIX_MICRO);

 nano: Prefix -> Prefix
 nano(x) == scale_prefix(x, PREFIX_NANO );

 pico: Prefix -> Prefix
 pico(x) == scale_prefix(x, PREFIX_PICO ); 
 
------------------------------------------------------------------------------------
--@section SI common time measurements, which are the same across SI and BIS etc.
------------------------------------------------------------------------------------

 second: ConversionSchema * UnitSystem -> Second
 second(cs, u) == mk_MeasurementSystem(UNIT_TIME, cs, u);
 
 --@doc we take hour as the "base" unit of time, instead of SI's second given our use (i.e. smaller numbers?);
 --   this can be easily generalised to seconds or whatver needed 
 minute: ConversionSchema * UnitSystem -> Second
 minute(cs, u) == scaleMS(60, second(cs, u));
 
 hour: ConversionSchema * UnitSystem -> Second 
 hour(cs, u) == scaleMS(60, minute(cs, u));
 
 day: ConversionSchema * UnitSystem -> Second 
 day(cs, u) == scaleMS(24, hour(cs, u));
 
 week: ConversionSchema * UnitSystem -> Second
 week(cs, u) == scaleMS(DAYS_PER_WEEK, day(cs, u));
  
-----------------------------------------------------------------------------------
--@section SI accepted constants and known dimensions
------------------------------------------------------------------------------------
values
 METER   : Meter    = mk_MeasurementSystem(UNIT_LENGTH   , SI, SI_UNIT);
 KILOGRAM: Kilogram = mk_MeasurementSystem(UNIT_MASS     , SI, SI_UNIT);
 SECOND : Second   = mk_MeasurementSystem(UNIT_TIME     , SI, SI_UNIT); 
 KELVIN  : Kelvin   = mk_MeasurementSystem(UNIT_TEMP     , SI, SI_UNIT); 
        
 SI_FREQUENCY   : Frequency    = mk_MeasurementSystem(UNIT_FREQUENCY         , SI, SI_UNIT);
 SI_VELOCITY    : Velocity     = mk_MeasurementSystem(UNIT_VELOCITY          , SI, SI_UNIT);
 
functions
 
 --@doc transform given measurement system to the International Standard system     
 metrify: MeasurementSystem -> Quantity
 metrify(ms) == ms_conv(ms, SI).quantity;
 
 SI_ASTRONOMICAL: () -> Meter
 SI_ASTRONOMICAL() == scaleMS(149597870700, METER); 
 
 SI_TONNE: () -> Kilogram
 SI_TONNE() == scaleMS(mag(kilo(1)), KILOGRAM);
        
------------------------------------------------------------------------------------
--@section SI common imperial conversions
------------------------------------------------------------------------------------

 SI_YARD: () -> Meter
 SI_YARD() == scaleMS(0.9144, METER);
 
 SI_FOOT: () -> Meter
 SI_FOOT() == scaleMS(1/3, SI_YARD());
 
 SI_INCH: () -> Meter
 SI_INCH() == scaleMS(1/36, SI_YARD());
 
 SI_MILE: () -> Meter
 SI_MILE() == scaleMS(1760, SI_YARD());
 
 SI_NAUTICAL_MILE: () -> Meter
 SI_NAUTICAL_MILE() == scaleMS(1852, METER);
 
 SI_POUND: () -> Kilogram
 SI_POUND() == scaleMS(0.45359237, KILOGRAM);
 
 SI_OUNCE: () -> Kilogram
 SI_OUNCE() == scaleMS(1/16, SI_POUND());
 
 SI_STONE: () -> Kilogram
 SI_STONE() == scaleMS(14, SI_POUND());

 SI_KNOT: () -> Velocity
 SI_KNOT() == ms_div(SI_NAUTICAL_MILE(), hour(SI, SI_UNIT));
 
 SI_MPH: () -> Velocity
 SI_MPH() == ms_div(SI_MILE(), hour(SI, SI_UNIT));
 
 --@doc the standard velocity in SI is meters per second 
 SI_MPS: () -> Velocity
 SI_MPS() == SI_VELOCITY;

 SI_KPH: () -> Velocity
 SI_KPH() == ms_div(kilo(METER), hour(SI, SI_UNIT));
 
 SI_DEGREES_FARENHEIT: Magnitude -> Kelvin
 SI_DEGREES_FARENHEIT(x) == scaleMS((x + 459.67) * 5/9, KELVIN);
      
------------------------------------------------------------------------------------
--@section British Imperial measurement system setup and common conversions
------------------------------------------------------------------------------------
values
 BIS_UNIT   : UnitSystem = "BIS";
 --@doc British Imperial System; choose Rankine instead of Farenheit for offset simplicity
 BIS: ConversionSchema = CONV_ID ++ 
  { <Length> |-> 0.9143993, <Mass> |-> 0.453592338, <Temperature> |-> 5/9  };

types
 --@doc an BIS measurement system has an BIS conversion
 BIS_MeasurementSystem = MeasurementSystem
 inv ms == ms.schema = BIS and ms.unit = BIS_UNIT;
 
 Yard = BIS_MeasurementSystem
 inv ms == ms.quantity.dim = DLENGTH;
 
 Pound = BIS_MeasurementSystem
 inv ms == ms.quantity.dim = DMASS;
 
 Rankine = BIS_MeasurementSystem
 inv ms == ms.quantity.dim = DTEMP;
 
 BISVelocity = BIS_MeasurementSystem
 inv ms == ms.quantity.dim = DVELOCITY;
 
values
 BIS_YARD    : Yard       = mk_MeasurementSystem(UNIT_LENGTH  , BIS, BIS_UNIT);
 BIS_POUND   : Pound      = mk_MeasurementSystem(UNIT_MASS    , BIS, BIS_UNIT);
 BIS_RANKINE : Rankine    = mk_MeasurementSystem(UNIT_TEMP    , BIS, BIS_UNIT);
  BIS_VELOCITY: BISVelocity= mk_MeasurementSystem(UNIT_VELOCITY, BIS, BIS_UNIT);
functions
 
 --@doc transform given measurement system to the British Imperial System
 imperialise: MeasurementSystem -> Quantity
 imperialise(ms) == ms_conv(ms, BIS).quantity;

 BIS_FOOT: () -> Yard
 BIS_FOOT() == scaleMS(1/3, BIS_YARD);

 BIS_INCH: () -> Yard
 BIS_INCH() == scaleMS(1/12, BIS_FOOT());
 
 BIS_MILE: () -> Yard
 BIS_MILE() == scaleMS(1760, BIS_YARD);

 BIS_OUNCE: () -> Pound
 BIS_OUNCE() == scaleMS(1/12, BIS_POUND);
 
 BIS_MILE_PER_HOUR: () -> BISVelocity
 BIS_MILE_PER_HOUR() == ms_div(BIS_MILE(), hour(BIS, BIS_UNIT));
        
------------------------------------------------------------------------------------
--@section Other measurement system of interest setup
------------------------------------------------------------------------------------
values 
 CGS_UNIT: UnitSystem = "CGS";
 MHC_UNIT: UnitSystem = "MHC";
 
 
 --@doc Centemetre-Gram-Second system; 
 CGS: ConversionSchema = CONV_ID ++ { <Length> |-> mag(centi(METER)), <Mass> |-> mag(milli(KILOGRAM)) };

 --@doc Miligram-Hour system;
 MHC: ConversionSchema = CONV_ID ++ { <Mass> |-> mag(milli(milli(KILOGRAM))), <Time> |-> mag(hour(SI, SI_UNIT)), <Temperature> |-> -272.15 };
 
 WEEKS_PER_YEAR : nat1 = 52;
 MONTHS_PER_YEAR: nat1 = 12;
 DAYS_PER_YEAR  : nat1 = 365;

types
 CGS_MeasurementSystem = MeasurementSystem
 inv ms == ms.schema = CGS and ms.unit = CGS_UNIT;
 
 Centimetre = CGS_MeasurementSystem
 inv ms == ms.quantity.dim = DLENGTH;
 
 Gram = CGS_MeasurementSystem
 inv ms == ms.quantity.dim = DMASS;
 
 MHC_MeasurementSystem = MeasurementSystem
 inv ms == ms.schema = MHC and ms.unit = MHC_UNIT;
 
 Milligram = MHC_MeasurementSystem
 inv ms == ms.quantity.dim = DMASS;
 
 Hour = MHC_MeasurementSystem
 inv ms == ms.quantity.dim = DTIME;
 
 Celcius = MHC_MeasurementSystem
 inv ms == ms.quantity.dim = DTEMP;
  
 WhichMonth = nat1
 inv wm == wm <= MONTHS_PER_YEAR;
 
 HowManyDays = nat1
 inv hmd == hmd in set {28,...,31};
    
values 
 CENTIMETRE: Centimetre = mk_MeasurementSystem(UNIT_LENGTH, CGS, CGS_UNIT);
 GRAM      : Gram    = mk_MeasurementSystem(UNIT_MASS  , CGS, CGS_UNIT);

 MGRAM  : Milligram = mk_MeasurementSystem(UNIT_MASS, MHC, MHC_UNIT); --1mg=(1kg/1000)/1000
 MHOUR   : Hour    = mk_MeasurementSystem(UNIT_TIME, MHC, MHC_UNIT);--hour(MHC); --1hr
 MCELCIUS: Celcius  = mk_MeasurementSystem(UNIT_TEMP, MHC, MHC_UNIT); --1c = -272.15k
 --MMHG  : IntQuantity   = UNIT_????; in KILOGRAM / (METER * SECOND**2)

 DAYS_PER_MONTH : map WhichMonth to HowManyDays = {1|->31, 2|->28, 3|->31, 4|->30, 5|->31, 6|->30,
                                                7|->31, 8|->31, 9|->30, 10|->31, 11|->30, 12|->31};

functions
 
 hDAY: () -> Hour
 hDAY() == scaleMS(HOURS_PER_DAY, MHOUR);
 
 hWEEK : () -> Hour 
 hWEEK() == scaleMS(DAYS_PER_WEEK, hDAY());
 
 hMONTH: WhichMonth -> Hour 
 hMONTH(m) == scaleMS(DAYS_PER_MONTH(m), hDAY());
 
 hAVG_MONTH: () -> Hour
 hAVG_MONTH() == scaleMS(averages_r(rng DAYS_PER_MONTH), hDAY());
 
 hYEAR : () -> Hour
 hYEAR() == scaleMS(DAYS_PER_YEAR, hDAY());
 
 hW2M: () -> Hour
 hW2M() == ms_times(hWEEK(), ms_inv(hAVG_MONTH()));
 
 weeks2months_hrs: nat1 -> Hour
 weeks2months_hrs(w) == scaleMS(w, hW2M());
 
 --@doc 3 times a day = DAY() / 3 = 1 times 8 hours;
 --   DAY()= 24hrs / 3 =  8hrs
 --@doc don't change the dimension vector, so just scale by the inverse
 n_times_day2every_x_hours: nat1 -> real
 n_times_day2every_x_hours(times_a_day) == mag(scaleMS(1/mag(scaleMS(times_a_day, MHOUR)), hDAY()));
    
values
 MDC_UNIT: UnitSystem = "MDC";
 MDC: ConversionSchema = MHC ++ { <Time> |-> mag(metrify(hDAY())) };
 
 MWC_UNIT: UnitSystem = "MWC";
 MWC: ConversionSchema = MHC ++ { <Time> |-> mag(metrify(hWEEK())) };
  
types
 MDC_MeasurementSystem = MeasurementSystem
 inv ms == ms.schema = MDC and ms.unit = MDC_UNIT;
 
 Day = MDC_MeasurementSystem
 inv ms == ms.quantity.dim = DTIME;
 
 MWC_MeasurementSystem = MeasurementSystem
 inv ms == ms.schema = MWC and ms.unit = MWC_UNIT;
 
 Week = MWC_MeasurementSystem
 inv ms == ms.quantity.dim = DTIME;

values
 MDAY : Day  = mk_MeasurementSystem(UNIT_TIME, MDC, MDC_UNIT);
 MWEEK: Week = mk_MeasurementSystem(UNIT_TIME, MWC, MWC_UNIT);
  
functions
 
 dWEEK: () -> Day
 dWEEK() == scaleMS(DAYS_PER_WEEK, MDAY);
 
 weeks2days: nat1 -> Day
 weeks2days(n) == scaleMS(n, dWEEK());

 wDAY: () -> Week
 wDAY() == scaleMS(1/DAYS_PER_WEEK, MWEEK);
 
 wAVG_MONTH: () -> Week
 wAVG_MONTH() == scaleMS(averages_r(rng DAYS_PER_MONTH), wDAY());
 
 wW2M: () -> Week
 wW2M() == ms_times(MWEEK, ms_inv(wAVG_MONTH()));
 
 weeks2months_wks: nat1 -> Week
 weeks2months_wks(w) == scaleMS(w, wW2M());

------------------------------------------------------------------------------------
--@section SI constants
------------------------------------------------------------------------------------  

values
 HERTZ    : Hertz = SI_FREQUENCY;
 CAESIUM_FREQUENCY: Hertz = scaleMS(9192631770, HERTZ);
 SPEED_OF_LIGHT   : Velocity = scaleMS(299792458, SI_VELOCITY);
 
------------------------------------------------------------------------------------
--@section Useful for testing in approximating
------------------------------------------------------------------------------------  
values
 ORDER_MAGNITUDE: nat = 10;
 
functions
 
 --@doc approximate to the given precision; useful for comparing up to an order of magnitude
 approx: Magnitude * nat -> Magnitude
 approx(m, precision) == m * (10**precision);
 
 --@doc approximate  both sides to the same precision then floor them for equality testing
 approx_eq: Magnitude * Magnitude * nat -> bool
 approx_eq(lhs, rhs, precision) == 
  let 
   lhs': Magnitude = approx(lhs, precision),
   rhs': Magnitude = approx(rhs, precision)
   in
    floor(lhs') = floor(rhs');

------------------------------------------------------------------------------------
--@section Useful conversions
------------------------------------------------------------------------------------
 
 --@doc miles per hour in BIS to metre per second in SI
 --@doc if want kilometres per hour, need to create a new MeasurementSystem on KPH
 mph2mps: Magnitude -> Magnitude
 mph2mps(mph) == mag(metrify(scaleMS(mph, BIS_MILE_PER_HOUR())));
 
------------------------------------------------------------------------------------
--@section Theorems between measurement systems from Isabelle proofs for sense check
------------------------------------------------------------------------------------
 refl_bool: bool -> bool
 refl_bool(x) == x; 

 MEASUREMENT_SYSTEMS: () -> set1 of MeasurementSystem 
 MEASUREMENT_SYSTEMS() ==
    {SI_ASTRONOMICAL(), SI_TONNE(), SI_YARD(), SI_FOOT(), 
     SI_INCH(), SI_MILE(), SI_NAUTICAL_MILE(), SI_POUND(), SI_OUNCE(), SI_STONE(), 
     SI_KNOT(), SI_MPH(), SI_MPS(), SI_KPH(), BIS_FOOT(), 
     BIS_INCH(), BIS_MILE(), BIS_OUNCE(), 
     BIS_MILE_PER_HOUR(), hDAY(), hWEEK(), hAVG_MONTH(), hYEAR(), wDAY(), wAVG_MONTH()
    };
    
 CONVERSION_SCHEMAS : () -> set1 of ConversionSchema
 CONVERSION_SCHEMAS() == {SI, BIS, MHC, MDC, MWC};
 
 DIMENSION_VECTORS  : () -> set1 of DimensionVector  
 DIMENSION_VECTORS() == 
    {DLENGTH, DMASS, DTIME, DTEMP, 
    DFREQUENCY, DVELOCITY
    };
    
 MAGNITUDES: () -> set1 of Magnitude
 MAGNITUDES() == { 10, 20, 30, 40, 100, centi(100), kilo(10), pico(1), deci(10) };

traces
------------------------------------------------------------------------------------
--@todo
--@subsection Measurement equiv and comparison lemmas 
------------------------------------------------------------------------------------
------------------------------------------------------------------------------------
--@todo
--@subsection Measurement equiv and comparison lemmas (ISQProof.thy, ISQQuantities.thy)
------------------------------------------------------------------------------------
------------------------------------------------------------------------------------
--@subsection Conversion lemmas (ISQConversion.thy)
------------------------------------------------------------------------------------

 --@doc qconv cid x = x; 32 tests
 ISQ_C_id: let ms in set MEASUREMENT_SYSTEMS() in
     (refl_bool(ms_quant_conv(CONV_ID, ms) = ms));
 
 --@doc qconv (c1 o c2) x = qconv c1 (qconv c2 x); 640 tests
 ISQ_C_comp: let ms in set MEASUREMENT_SYSTEMS() in 
        let c1 in set CONVERSION_SCHEMAS() in
         let c2 in set CONVERSION_SCHEMAS() \ {c1} in 
          refl_bool(
          ms_quant_conv(conv_comp(c1, c2), ms) 
          = 
          ms_quant_conv(c1, ms_quant_conv(c2, ms))
          );

 --@doc qconv (1/c o c) x = x; 160 tests         
 ISQ_C_inv: let ms in set MEASUREMENT_SYSTEMS() in 
        let c in set CONVERSION_SCHEMAS() in
          refl_bool(
           ms_quant_conv(conv_inv(c), ms_quant_conv(c, ms)) 
           = 
           ms
          );

 --@doc 1280 tests         
 ISQ_C_scalQ: let ms in set MEASUREMENT_SYSTEMS() in 
          let c in set CONVERSION_SCHEMAS() in
            let d in set MAGNITUDES() in
            refl_bool(
             ms_quant_conv(c, scaleMS(d, ms)) 
             = 
             scaleMS(d, ms_quant_conv(c, ms))
            );
          
------------------------------------------------------------------------------------
--@todo
--@subsection Various other SI lemmas (SI_Constants.thy, SI_Derived, SI_Accepted, 
------------------------------------------------------------------------------------
------------------------------------------------------------------------------------
--@subsection SI_Accepted
------------------------------------------------------------------------------------
 --@doc 1 hour = 3600 sec; 1 test
 SIA_1: let
      lhs: Magnitude = mag(hour(SI, SI_UNIT)),
      rhs: Magnitude = mag(scaleMS(3600, second(SI, SI_UNIT)))
     in
      refl_bool(lhs = rhs);
  --@doc 25 m/s = 90 km/hr; 1 test
 SIA_2: let
      lhs: Magnitude = mag(scaleMS(25, SI_MPS())),
      rhs: Magnitude = mag(scaleMS(90, SI_KPH()))
     in
      refl_bool(lhs = rhs);
            
------------------------------------------------------------------------------------
--@subsection SI_Prefix
------------------------------------------------------------------------------------
 --@doc 2.3 (centi *q metre)**3 = pico(2.3) * metre^3; 1 test
 SIP_1: let
      lhs: Magnitude = mag(scaleMS(2.3, ms_itself_n(centi(METER),3))),
      rhs: Magnitude = mag(scaleMS(pico(2.3), ms_itself_n(METER, 3)))
     in
      refl_bool(approx_eq(lhs, rhs, ORDER_MAGNITUDE));
 --@doc 1 (centi *q metre)**-1 = 100 * metre**-1
 SIP_2: let
      lhs: Magnitude = mag(ms_inv(centi(METER))),
      rhs: Magnitude = mag(scaleMS(100, ms_inv(METER)))
     in
      refl_bool(approx_eq(lhs, rhs, ORDER_MAGNITUDE));
                    
------------------------------------------------------------------------------------
--@subsection SI Imperial 
------------------------------------------------------------------------------------
 --@doc 1 mile = 5280 feet in SI's view up to some order of magnitude but not others
 SIBIS_1: let
       lhs: Magnitude = mag(SI_MILE()),
       rhs: Magnitude = mag(scaleMS(5280, SI_FOOT()))
      in
       refl_bool(approx_eq(lhs, rhs, 2) and not approx_eq(lhs, rhs, 3));
 --@doc 1 mph = 1.609344 kph in SI; but if we want more generally, create a MS for KPH (like CGS)
 SIBIS_2: let
        lhs: Magnitude = mag(SI_MPH()),
       rhs: Magnitude = mag(scaleMS(1.609344, SI_KPH()))
      in
       refl_bool(approx_eq(lhs, rhs, ORDER_MAGNITUDE));         
------------------------------------------------------------------------------------
--@subsection BIS 
------------------------------------------------------------------------------------
 --@doc 1yard ~= 0.914 metres
 BIS_0: refl_bool(mag(metrify(BIS_YARD)) = 0.9143993);

 --@doc lhs = magnitude of a BIS foot in SI,
 --   rhs = magnitude of 1/3 of BIS yard in SI
 --   test= metrify(1 foot) = metrify(1 yard) / (3 * metre)
 BIS_1: let
     lhs: Magnitude = mag(metrify(BIS_FOOT())),
     rhs: Magnitude = mag(metrify(BIS_YARD).mag) / (3 * mag(METER))
    in   
     refl_bool(approx_eq(lhs, rhs, ORDER_MAGNITUDE));

 --@doc lhs = magnitude of a BIS 70mph in SI metre per second
 --   rhs = magnitude of particular factor in SI metre per second
 BIS_2: let
     lhs: Magnitude = mph2mps(70),
     rhs: Magnitude = mag(scaleMS((704087461 / 22500000), SI_MPS()))
    in   
     refl_bool(approx_eq(lhs, rhs, ORDER_MAGNITUDE));

 --@doc lhs = magnitude of 1 centimetre in CGS as a BIS yard
 --   rhs = magnitude of BIS yard by given factor 
 BIS_3: let
     lhs: Magnitude = mag(imperialise(scaleMS(1, CENTIMETRE))), 
     rhs: Magnitude = mag(scaleMS(100000 / 9143993, BIS_YARD))
    in   
     refl_bool(approx_eq(lhs, rhs, ORDER_MAGNITUDE));
                        
------------------------------------------------------------------------------------
--@subsection CGS
------------------------------------------------------------------------------------
 --@doc 1KM in SI = 100000 centimetres in CGS
 CGS_1: let 
      lhs: Magnitude = mag(ms_conv(/*SI, CGS, kilo(METER)*/kilo(METER), CGS)),
      rhs: Magnitude = mag(scaleMS(100000, CENTIMETRE))
     in
      refl_bool((lhs = 100000) and (lhs = rhs));

 --@doc 100 cm in CGS = 1 metre in SI
 CGS_2: refl_bool(mag(metrify(scaleMS(100, CENTIMETRE))) = mag(METER));

------------------------------------------------------------------------------------
--@subsection MHC, MDC, MWC
------------------------------------------------------------------------------------
 --@doc (in hours) week / avg_month = week / year * avg_month / year
 MHC_1: let
      lhs: MeasurementSystem = hW2M(),
      rhs: MeasurementSystem = ms_div(ms_div(hWEEK(), hYEAR()), ms_div(hAVG_MONTH(), hYEAR()))
     in
      refl_bool(lhs = rhs);

 --@doc 4 weeks = 0.9438202247191011 avg_month
 MWC_1: let
      lhs: MeasurementSystem = weeks2months_wks(4),
      rhs: MeasurementSystem = scaleMS(0.9438202247191011, wAVG_MONTH())
     in
      refl_bool(lhs = rhs);
 
 --@doc 12 weeks = 84 days
 MWC_2: let
      lhs: MeasurementSystem = scaleMS(12, MWEEK),
      rhs: MeasurementSystem = scaleMS(84, MDAY)
     in
      refl_bool(mag(ms_conv(lhs, MDC)) = mag(rhs));     
------------------------------------------------------------------------------------
--@subsection Other; total 2132 tests
------------------------------------------------------------------------------------

 --@doc 1KM = 1093.614 yards
 T1: refl_bool(imperialise(kilo(METER)).mag = 1093.6141355313812);

 --@doc 1yard = 0.914 metres
 T2: refl_bool(metrify(BIS_YARD).mag = 0.9143993);


end VELOCITY
\end{vdm_al}
\end{document}
